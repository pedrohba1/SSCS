\documentclass[12pt, %
openright, 
oneside, %
%twoside, %TCC: Se seu texto tem mais de 100 páginas, descomente esta linha e comente a anterior
a4paper,    %
%english,   %
brazil]{facom-ufu-abntex2}

\usepackage{graphicx}
\graphicspath{{figuras/}{pictures/}{images/}{./}} % where to search for the images

\newcommand{\blue}[1]{\textcolor{blue}{#1}}
\newcommand{\red}[1]{\textcolor{red}{#1}}


\autor{Pedro Henrique Bufulin de Almeida} %TCC
\data{2021}
\orientador{Pedro Frosi Rosa} %TCC
%\coorientador{Algum?} %TCC

% ---
% Informações de dados para CAPA e FOLHA DE ROSTO
% ---
 
\titulo{Sistema de segurança doméstica: uma solução open source e com privacidade} %TCC

\hypersetup{pdfkeywords={palavra 1}{palavra 2}{palavra 4}{palavra 4}{palavra 5}} %TCC

\begin{document} 
\frenchspacing 

% ----------------------------------------------------------
% ELEMENTOS PRÉ-TEXTUAIS
% ----------------------------------------------------------
%\pretextual
\imprimircapa
\imprimirfolhaderosto


% ---
% Inserir folha de aprovação
% ---
%
% \includepdf{folhadeaprovacao_final.pdf} %TCC: depois de aprovado o trabalho, descomente esta linha e comente o próximo bloco para incluir scan da folha de aprovação.
%
\begin{folhadeaprovacao}

  \begin{center}
    {\ABNTEXchapterfont\large\imprimirautor}

    \vspace*{\fill}\vspace*{\fill}
    {\ABNTEXchapterfont\bfseries\Large\imprimirtitulo}
    \vspace*{\fill}
    
    \hspace{.45\textwidth}
    \begin{minipage}{.5\textwidth}
        \imprimirpreambulo
    \end{minipage}%
    \vspace*{\fill}
   \end{center}
    
   Trabalho aprovado. \imprimirlocal, 01 de novembro de 2016: %TCC:

   \assinatura{\textbf{\imprimirorientador} \\ Orientador}  
   \assinatura{\textbf{Professor}}% \\ Convidado 1} %TCC:
   \assinatura{\textbf{Professor}}% \\ Convidado 2} %TCC:
   %\assinatura{\textbf{Professor} \\ Convidado 3}
   %\assinatura{\textbf{Professor} \\ Convidado 4}
      
   \begin{center}
    \vspace*{0.5cm}
    {\large\imprimirlocal}
    \par
    {\large\imprimirdata}
    \vspace*{1cm}
  \end{center}
  
\end{folhadeaprovacao}
% ---


%%As seções dedicatória, agradecimento e epígrafe não são obrigatórias.
%%Só as mantenha se achar pertinente.

% ---
% Dedicatória
% ---
%\begin{dedicatoria}
%   \vspace*{\fill}
%   \centering
%   \noindent
%   \textit{Dedico a \lipsum[10]}  %TCC:
%   \vspace*{\fill}
%\end{dedicatoria}
% ---

% ---
% Agradecimentos
% ---
%\begin{agradecimentos}
%Agradeço a \lipsum[30]. %TCC:
%\end{agradecimentos}
% ---

% ---
% Epígrafe
% ---
%\begin{epigrafe}
%    \vspace*{\fill}
%	\begin{flushright}
%		\textit{``Alguma citação que ache conveniente? \lipsum[10]''} %TCC:
%	\end{flushright}
%\end{epigrafe}
% ---



\begin{resumo} %TCC:
 Os sistemas de segurança domésticos carecem de soluções
 que sejam customizáveis de acordo com a falha de segurança que existe em cada domicílio.
 Frequentemente encontra-se no mercado câmeras que fazem apenas a gravação com baixa resolução, e
 caso busque algo mais sofisticado, existem modelos envolvendo reconhecimento facial, mas eles são ainda mais caros e raros.
 Além disso, as soluções de segurança pública ou providas por empresas privadas, quando existem, podem usar as informações 
 coletadas para seus próprios benefícios, diminuindo a privacidade do indivíduo. Este trabalho propõe uma solução que seja
 fácil de implementar, com ferramentas prontas para uso e customizável por ser de código aberto,  
 trazendo com esta última característica o poder para qualquer um de criar a sua própria segurança 
 e modificar este projeto de acordo com suas necessidades. 

 \vspace{\onelineskip}
    
 \noindent
 \textbf{Palavras-chave}: Segurança, código,  aberto, reconhecimento, facial, câmera.  %TCC:
\end{resumo}

% ---
% inserir lista de ilustrações
% ---
\pdfbookmark[0]{\listfigurename}{lof}
\listoffigures*
\cleardoublepage
% ---

% ---
% inserir lista de tabelas
% ---
\pdfbookmark[0]{\listtablename}{lot}
\listoftables*
\cleardoublepage
% ---



% ---
% inserir lista de abreviaturas e siglas
% ---
\begin{siglas} %TCC:
  \item[Fig.] Area of the $i^{th}$ component
  \item[456] Isto é um número
  \item[123] Isto é outro número
  \item[Zézão] este é o meu nome
  \item[CCTV] Close-circuit television camera 
\end{siglas}
% ---

%% ---
%% inserir lista de símbolos, se for adequado ao trabalho. %TCC:
%% ---
%\begin{simbolos}
%  \item[$ \Gamma $] Letra grega Gama
%  \item[$ \Lambda $] Lambda
%  \item[$ \zeta $] Letra grega minúscula zeta
%  \item[$ \in $] Pertence
%\end{simbolos}
%% ---

% ---
% inserir o sumario
% ---
\pdfbookmark[0]{\contentsname}{toc}
\tableofcontents*
\cleardoublepage
% ---





% ----------------------------------------------------------
% ELEMENTOS TEXTUAIS
% ----------------------------------------------------------
\textual


% ----------------------------------------------------------
% Introdução
% ----------------------------------------------------------

\chapter[Introdução]{Introdução}
%TCC:
% Contextualização, problema, hipótese, objetivo geral, objetivos específicos, justificativa e resultados esperados.
Em 2019 foi estimado que existem 200 milhões de câmeras de vigilância na China. Na última década,
avanços tecnológicos tornaram essas câmeras ainda mais eficientes em monitorar 1.4 bilhões de chineses. 
O reconhecimento de rostos por câmeras começou a ser uma realidade em 2010 quando pesquisadores 
descobriram algoritmos de deep learning usados para reconhecer imagens e voz. Esses algoritmos podem
também inferir em tempo real a quantidade e a densidade de pessoas numa dada imagem
\cite{qiang2019road}. É notável que esse nível de vigilância está se tornando uma realidade em muitos outros
países no mundo além da China. Nesse sentido, é perceptível que existe uma demanda por aumentar a segurança usando os meios 
necessários, mas sem que isso signifique diminuir a privacidade dos indivíduos colocando suas informações disponíveis aos 
governantes e corporações.

Em questão de proteção da privacidade, tecnologias de código aberto podem oferecer uma solução. Quando o Desenvolvimento
ocorre em aberto, é possível verificar diretamente se o proprietário está ativamente buscando segurança e privacidade e
entender como ele trata esses problemas. A possibilidade de estudar o processo seguido faz com que qualquer um possa
realizar uma auditoria independente. \cite{mardjan2016open}

Considerando-se a necessidade de criar uma solução que traga os benefícios da vigilância e mantendo a privacidade individual, surge
essa proposta da construção de um sistema de segurança doméstica que oferece as opções de reconhecimento de imagem
por inteligência artificial, além da visualização em tempo real da gravação, e que tenha tanto o código fonte quanto o hardware
abertos em virtude da transparência.

\section{Objetivos}

O objetivo principal deste trabalho é performar a concepção, modelagem e implementação de um sistema de segurança doméstica utilizando
apenas hardware e software que sejam abertos, assim como o projeto resultante final deverá ser. Pretende-se que seja possível visualizar 
a imagem da câmera em tempo real, adicionar rostos de pessoas que sejam tidos como confiáveis e não passíveis de alarme, enviar notificações
para o usuário quando for identificado alguém na câmera que não passar como confiável. Também será disponibilizada uma documentação que 
contém as instruções para implementar o sistema em domicílio, para que qualquer indivíduo provido do equipamento possa colocá-lo em uso.

\section{Método}

Para os objetivos apresentados Seção 1.1 serem alcançados, os seguintes passos serão seguidos para o desenvolvimento deste sistema:

\begin{itemize}
    \item  Criar um protótipo de solução para a questão apresentada:
    \begin{itemize}
      \item Definir a arquitetura do sistema, tanto do backend quanto do frontend.
      \item Definir o tipo de API e qual a melhor linguagem para criá-la.
      \item definir qual o hardware a ser utilizado, tanto da câmera quanto do sistema que processará as informações da câmera. 
      \item Explorar as capacidades do hardware da câmera e do sistema que coordena suas ações, além de encontrar suas limitações e como contorná-las.
      \item Se for necessário, definir como implementar um servidor para que seja processada e armazenada a imagem da câmera.
  \end{itemize}
\end{itemize}

\begin{itemize}
  \item  Programar o protótipo que solucionará o problema apresentado:
  \begin{itemize}
    \item Escolher algoritmos de compressão de imagem adequado para as gravações.
    \item Implementar a aplicação de acesso do usuário.
    \item Implementar o software que será responsável pelo reconhecimento de imagem e streaming do vídeo para o aplicativo.
  \end{itemize}
\end{itemize}




\chapter{Fundamentação Teórica}


Neste capítulo estão as tecnologias e os conceitos que serão utilizados no decorrer da construção deste projeto.
São conceitos relacionados à arquitetura do sistema que será construído, as abordagens que já existem, \emph{hardware}, 
linguagens e \emph{frameworks} que serão utilizados.


\section{Arquitetura de transmissão em tempo real}
 
A figura abaixo mostra um sistema de cliente-servidor para trocar de dados de multimídia.
Na origem, tem-se a gravação comprimida, codificada e armazenada no dispositivo de armazenamento,
como um disco rígido, por exemplo. Em seguida, por meio de algum software de tratamento de mídia que será produzido,
esses arquivos são requisitados por usuários e entregues de acordo. Um protocolo de transferência é utilizado para
entregar os dados de multimídia para o cliente (tais como RTMP ou HLS), onde eles são primeiramente armazenados
na memória principal e eventualmente decodificados e apresentados ao usuário.


\begin{figure}[!ht]
  \centering
\includegraphics[width=1\linewidth]{Capturar.PNG}
\caption[Representação de um arquitetura de tempo real genérica]{Arquitetura genérica de transmissão em tempo real}
\label{fig:graficosVariandoTamanhoRede}
\end{figure}

Esta arquitetura é a que levarei em consideração na implementação do projeto. O servidor que será implementado utilizando
um microcomputador contem a câmera e sistema operacional para realizar o processamento e transmissão dos dados de 
multimídia.



\chapter{Revisão Bibliográfica}
%TCC:
Um ou mais capítulos (por exemplo, se há duas linhas de trabalhos relacionados).



\chapter{Desenvolvimento}
%TCC:
Um ou mais capítulos (por exemplo um para testes)


\begin{figure}[!ht]
    \centering
	\includegraphics[width=0.55\linewidth]{imagemExemplo.pdf}
	\caption[Isso é o que aparece no sumário]{Imagem de exemplo.}
	\label{fig:graficosVariandoTamanhoRede}
\end{figure}


%TCC:
%TCC:
%TCC:
%TCC:

% ---
% Conclusão
% ---
\chapter[Conclusão]{Conclusão}
%TCC:
E daí?





% ----------------------------------------------------------
% ELEMENTOS PÓS-TEXTUAIS
% ----------------------------------------------------------
\postextual


% ----------------------------------------------------------
% Referências bibliográficas
% ----------------------------------------------------------
\bibliography{abntex2-modelo-references}


%% ----------------------------------------------------------
%% Apêndices TCC: só mantenha se for pertinente.
%% ----------------------------------------------------------

% ---
% Inicia os apêndices
% ---
\begin{apendicesenv}

% Imprime uma página indicando o início dos apêndices
\partapendices

% ----------------------------------------------------------
\chapter{Quisque libero justo}
% ----------------------------------------------------------

\lipsum[50]

% ----------------------------------------------------------
\chapter{Coisas que fiz e que achei interessante mas não tanto para entrar no corpo do texto}
% ----------------------------------------------------------
\lipsum[55-57]

\end{apendicesenv}
% ---


% ----------------------------------------------------------
% Anexos %TCC: so mantenha se pertinente.
% ----------------------------------------------------------

% ---
% Inicia os anexos
% ---
\begin{anexosenv}

% Imprime uma página indicando o início dos anexos
\partanexos

% ---
\chapter{Eu sempre quis aprender latim}
% ---
\lipsum[30]

% ---
\chapter{Coisas que eu não fiz mas que achei interessante o suficiente para colocar aqui}
% ---

\lipsum[31]

% ---
\chapter{Fusce facilisis lacinia dui}
% ---

\lipsum[32]

\end{anexosenv}

%---------------------------------------------------------------------
% INDICE REMISSIVO
%---------------------------------------------------------------------

\printindex



\end{document}