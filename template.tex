\documentclass[12pt, %
openright, 
oneside, %
%twoside, %TCC: Se seu texto tem mais de 100 páginas, descomente esta linha e comente a anterior
a4paper,    %
%english,   %
brazil]{facom-ufu-abntex2}

\usepackage{graphicx}
\graphicspath{{figuras/}{pictures/}{images/}{./}} % where to search for the images

\newcommand{\blue}[1]{\textcolor{blue}{#1}}
\newcommand{\red}[1]{\textcolor{red}{#1}}


\autor{Pedro Henrique Bufulin de Almeida} %TCC
\data{2021}
\orientador{Pedro Frosi Rosa} %TCC
%\coorientador{Algum?} %TCC

% ---
% Informações de dados para CAPA e FOLHA DE ROSTO
% ---
 
\titulo{Sistema de segurança doméstica: uma solução open source e com privacidade} %TCC

\hypersetup{pdfkeywords={palavra 1}{palavra 2}{palavra 4}{palavra 4}{palavra 5}} %TCC

\begin{document} 
\frenchspacing 

% ----------------------------------------------------------
% ELEMENTOS PRÉ-TEXTUAIS
% ----------------------------------------------------------
%\pretextual
\imprimircapa
\imprimirfolhaderosto


% ---
% Inserir folha de aprovação
% ---
%
% \includepdf{folhadeaprovacao_final.pdf} %TCC: depois de aprovado o trabalho, descomente esta linha e comente o próximo bloco para incluir scan da folha de aprovação.
%
\begin{folhadeaprovacao}

  \begin{center}
    {\ABNTEXchapterfont\large\imprimirautor}

    \vspace*{\fill}\vspace*{\fill}
    {\ABNTEXchapterfont\bfseries\Large\imprimirtitulo}
    \vspace*{\fill}
    
    \hspace{.45\textwidth}
    \begin{minipage}{.5\textwidth}
        \imprimirpreambulo
    \end{minipage}%
    \vspace*{\fill}
   \end{center}
    
   Trabalho aprovado. \imprimirlocal, 01 de novembro de 2016: %TCC:

   \assinatura{\textbf{\imprimirorientador} \\ Orientador}  
   \assinatura{\textbf{Professor}}% \\ Convidado 1} %TCC:
   \assinatura{\textbf{Professor}}% \\ Convidado 2} %TCC:
   %\assinatura{\textbf{Professor} \\ Convidado 3}
   %\assinatura{\textbf{Professor} \\ Convidado 4}
      
   \begin{center}
    \vspace*{0.5cm}
    {\large\imprimirlocal}
    \par
    {\large\imprimirdata}
    \vspace*{1cm}
  \end{center}
  
\end{folhadeaprovacao}
% ---


%%As seções dedicatória, agradecimento e epígrafe não são obrigatórias.
%%Só as mantenha se achar pertinente.

% ---
% Dedicatória
% ---
%\begin{dedicatoria}
%   \vspace*{\fill}
%   \centering
%   \noindent
%   \textit{Dedico a \lipsum[10]}  %TCC:
%   \vspace*{\fill}
%\end{dedicatoria}
% ---

% ---
% Agradecimentos
% ---
%\begin{agradecimentos}
%Agradeço a \lipsum[30]. %TCC:
%\end{agradecimentos}
% ---

% ---
% Epígrafe
% ---
%\begin{epigrafe}
%    \vspace*{\fill}
%	\begin{flushright}
%		\textit{``Alguma citação que ache conveniente? \lipsum[10]''} %TCC:
%	\end{flushright}
%\end{epigrafe}
% ---



\begin{resumo} %TCC:
 Segundo a \citeonline[3.1-3.2]{NBR6028:2003}, o resumo deve ressaltar o
 objetivo, o método, os resultados e as conclusões do documento. A ordem e a extensão
 destes itens dependem do tipo de resumo (informativo ou indicativo) e do
 tratamento que cada item recebe no documento original. O resumo deve ser
 precedido da referência do documento, com exceção do resumo inserido no
 próprio documento. (\ldots) As palavras-chave devem figurar logo abaixo do
 resumo, antecedidas da expressão Palavras-chave:, separadas entre si por
 ponto e finalizadas também por ponto.

 \vspace{\onelineskip}
    
 \noindent
 \textbf{Palavras-chave}: Até, cinco, palavras-chave, separadas, por, vírgulas. %TCC:
\end{resumo}

% ---
% inserir lista de ilustrações
% ---
\pdfbookmark[0]{\listfigurename}{lof}
\listoffigures*
\cleardoublepage
% ---

% ---
% inserir lista de tabelas
% ---
\pdfbookmark[0]{\listtablename}{lot}
\listoftables*
\cleardoublepage
% ---



% ---
% inserir lista de abreviaturas e siglas
% ---
\begin{siglas} %TCC:
  \item[Fig.] Area of the $i^{th}$ component
  \item[456] Isto é um número
  \item[123] Isto é outro número
  \item[Zézão] este é o meu nome
  \item[CCTV] Close-circuit television camera 
\end{siglas}
% ---

%% ---
%% inserir lista de símbolos, se for adequado ao trabalho. %TCC:
%% ---
%\begin{simbolos}
%  \item[$ \Gamma $] Letra grega Gama
%  \item[$ \Lambda $] Lambda
%  \item[$ \zeta $] Letra grega minúscula zeta
%  \item[$ \in $] Pertence
%\end{simbolos}
%% ---

% ---
% inserir o sumario
% ---
\pdfbookmark[0]{\contentsname}{toc}
\tableofcontents*
\cleardoublepage
% ---





% ----------------------------------------------------------
% ELEMENTOS TEXTUAIS
% ----------------------------------------------------------
\textual


% ----------------------------------------------------------
% Introdução
% ----------------------------------------------------------

\chapter[Introdução]{Introdução}
%TCC:
% Contextualização, problema, hipótese, objetivo geral, objetivos específicos, justificativa e resultados esperados.
Em 2019 foi estimado que existem 200 milhões de câmeras de vigilância na China. Na última década,
avanços tecnológicos tornaram essas câmeras ainda mais eficientes em monitorar 1.4 bilhões de chineses. 
O reconhecimento de rostos por câmeras começou a ser uma realidade em 2010 quando pesquisadores 
descobriram algoritmos de deep learning usados para reconhecer imagens e voz. Esses algoritmos podem
também inferir em tempo real a quantidade e a densidade de pessoas numa dada imagem
\cite{qiang2019road}. É notável que esse nível de vigilância está se tornando uma realidade em muitos outros
países no mundo além da China. Nesse sentido, é perceptível que existe uma demanda por aumentar a segurança usando os meios 
necessários, mas sem que isso signifique diminuir a privacidade dos indivíduos colocando suas informações disponíveis aos 
governantes e corporações.

Em questão de proteção da privacidade, tecnologias de código aberto podem oferecer uma solução. Quando o Desenvolvimento
ocorre em aberto, é possível verificar diretamente se o proprietário está ativamente buscando segurança e privacidade e
entender como ele trata esses problemas. A possibilidade de estudar o processo seguido faz com que qualquer um possa
realizar uma auditoria independente. \cite{mardjan2016open}

Considerando-se a necessidade de criar uma solução que traga os benefícios da vigilância e mantendo a privacidade individual, surge
essa proposta da construção de um sistema de segurança doméstica que oferece as opções de reconhecimento de imagem
por inteligência artificial, além da visualização em tempo real da gravação, e que tenha tanto o código fonte quanto o hardware
abertos em virtude da transparência.

\section{Objetivos}

O objetivo principal deste trabalho é performar a concepção, modelagem e implementação de um sistema de segurança doméstica utilizando
apenas hardware e software que sejam abertos, assim como o projeto resultante final deverá ser. Pretende-se que seja possível visualizar 
a imagem da câmera em tempo real, adicionar rostos de pessoas que sejam tidos como confiáveis e não passíveis de alarme, enviar notificações
para o usuário quando for identificado alguém na câmera que não passar como confiável. Também será disponibilizada uma documentação que 
contém as instruções para implementar o sistema em domicílio, para que qualquer indivíduo provido do equipamento possa colocá-lo em uso.

\section{Método}

Para os objetivos apresentados Seção 1.1 serem alcançados, os seguintes passos serão seguidos para o desenvolvimento deste sistema

\begin{itemize}
  \item  First Level
  \begin{itemize}
    \item  Second Level
    \begin{itemize}
      \item  Third Level
      \begin{itemize}
        \item  Fourth Level
      \end{itemize}
    \end{itemize}
  \end{itemize}
\end{itemize}


\chapter{Revisão Bibliográfica}
%TCC:
Um ou mais capítulos (por exemplo, se há duas linhas de trabalhos relacionados).



\chapter{Desenvolvimento}
%TCC:
Um ou mais capítulos (por exemplo um para testes)


\begin{figure}[!ht]
    \centering
	\includegraphics[width=0.55\linewidth]{imagemExemplo.pdf}
	\caption[Isso é o que aparece no sumário]{Imagem de exemplo.}
	\label{fig:graficosVariandoTamanhoRede}
\end{figure}


%TCC:
%TCC:
%TCC:
%TCC:

% ---
% Conclusão
% ---
\chapter[Conclusão]{Conclusão}
%TCC:
E daí?





% ----------------------------------------------------------
% ELEMENTOS PÓS-TEXTUAIS
% ----------------------------------------------------------
\postextual


% ----------------------------------------------------------
% Referências bibliográficas
% ----------------------------------------------------------
\bibliography{abntex2-modelo-references}


%% ----------------------------------------------------------
%% Apêndices TCC: só mantenha se for pertinente.
%% ----------------------------------------------------------

% ---
% Inicia os apêndices
% ---
\begin{apendicesenv}

% Imprime uma página indicando o início dos apêndices
\partapendices

% ----------------------------------------------------------
\chapter{Quisque libero justo}
% ----------------------------------------------------------

\lipsum[50]

% ----------------------------------------------------------
\chapter{Coisas que fiz e que achei interessante mas não tanto para entrar no corpo do texto}
% ----------------------------------------------------------
\lipsum[55-57]

\end{apendicesenv}
% ---


% ----------------------------------------------------------
% Anexos %TCC: so mantenha se pertinente.
% ----------------------------------------------------------

% ---
% Inicia os anexos
% ---
\begin{anexosenv}

% Imprime uma página indicando o início dos anexos
\partanexos

% ---
\chapter{Eu sempre quis aprender latim}
% ---
\lipsum[30]

% ---
\chapter{Coisas que eu não fiz mas que achei interessante o suficiente para colocar aqui}
% ---

\lipsum[31]

% ---
\chapter{Fusce facilisis lacinia dui}
% ---

\lipsum[32]

\end{anexosenv}

%---------------------------------------------------------------------
% INDICE REMISSIVO
%---------------------------------------------------------------------

\printindex



\end{document}