\documentclass[12pt, %
openright, 
oneside, %
%twoside, %TCC: Se seu texto tem mais de 100 páginas, descomente esta linha e comente a anterior
a4paper,    %
%english,   %
brazil]{facom-ufu-abntex2}

\usepackage{graphicx}
\graphicspath{{figuras/}{pictures/}{images/}{./}} % where to search for the images

\newcommand{\blue}[1]{\textcolor{blue}{#1}}
\newcommand{\red}[1]{\textcolor{red}{#1}}


\autor{Pedro Henrique Bufulin de Almeida} %TCC
\data{2021}
\orientador{Pedro Frosi Rosa} %TCC
%\coorientador{Algum?} %TCC

% ---
% Informações de dados para CAPA e FOLHA DE ROSTO
% ---
 
\titulo{Sistema de segurança doméstica: uma solução open source e com privacidade} %TCC

\hypersetup{pdfkeywords={palavra 1}{palavra 2}{palavra 4}{palavra 4}{palavra 5}} %TCC

\begin{document} 
\frenchspacing 

% ----------------------------------------------------------
% ELEMENTOS PRÉ-TEXTUAIS
% ----------------------------------------------------------
%\pretextual
\imprimircapa
\imprimirfolhaderosto


% ---
% Inserir folha de aprovação
% ---
%
% \includepdf{folhadeaprovacao_final.pdf} %TCC: depois de aprovado o trabalho, descomente esta linha e comente o próximo bloco para incluir scan da folha de aprovação.
%
\begin{folhadeaprovacao}

  \begin{center}
    {\ABNTEXchapterfont\large\imprimirautor}

    \vspace*{\fill}\vspace*{\fill}
    {\ABNTEXchapterfont\bfseries\Large\imprimirtitulo}
    \vspace*{\fill}
    
    \hspace{.45\textwidth}
    \begin{minipage}{.5\textwidth}
        \imprimirpreambulo
    \end{minipage}%
    \vspace*{\fill}
   \end{center}
    
   Trabalho aprovado. \imprimirlocal, 01 de novembro de 2016: %TCC:

   \assinatura{\textbf{\imprimirorientador} \\ Orientador}  
   \assinatura{\textbf{Professor}}% \\ Convidado 1} %TCC:
   \assinatura{\textbf{Professor}}% \\ Convidado 2} %TCC:
   %\assinatura{\textbf{Professor} \\ Convidado 3}
   %\assinatura{\textbf{Professor} \\ Convidado 4}
      
   \begin{center}
    \vspace*{0.5cm}
    {\large\imprimirlocal}
    \par
    {\large\imprimirdata}
    \vspace*{1cm}
  \end{center}
  
\end{folhadeaprovacao}
% ---


%%As seções dedicatória, agradecimento e epígrafe não são obrigatórias.
%%Só as mantenha se achar pertinente.

% ---
% Dedicatória
% ---
%\begin{dedicatoria}
%   \vspace*{\fill}
%   \centering
%   \noindent
%   \textit{Dedico a \lipsum[10]}  %TCC:
%   \vspace*{\fill}
%\end{dedicatoria}
% ---

% ---
% Agradecimentos
% ---
%\begin{agradecimentos}
%Agradeço a \lipsum[30]. %TCC:
%\end{agradecimentos}
% ---

% ---
% Epígrafe
% ---
%\begin{epigrafe}
%    \vspace*{\fill}
%	\begin{flushright}
%		\textit{``Alguma citação que ache conveniente? \lipsum[10]''} %TCC:
%	\end{flushright}
%\end{epigrafe}
% ---



\begin{resumo} %TCC:
 Segundo a \citeonline[3.1-3.2]{NBR6028:2003}, o resumo deve ressaltar o
 objetivo, o método, os resultados e as conclusões do documento. A ordem e a extensão
 destes itens dependem do tipo de resumo (informativo ou indicativo) e do
 tratamento que cada item recebe no documento original. O resumo deve ser
 precedido da referência do documento, com exceção do resumo inserido no
 próprio documento. (\ldots) As palavras-chave devem figurar logo abaixo do
 resumo, antecedidas da expressão Palavras-chave:, separadas entre si por
 ponto e finalizadas também por ponto.

 \vspace{\onelineskip}
    
 \noindent
 \textbf{Palavras-chave}: Até, cinco, palavras-chave, separadas, por, vírgulas. %TCC:
\end{resumo}

% ---
% inserir lista de ilustrações
% ---
\pdfbookmark[0]{\listfigurename}{lof}
\listoffigures*
\cleardoublepage
% ---

% ---
% inserir lista de tabelas
% ---
\pdfbookmark[0]{\listtablename}{lot}
\listoftables*
\cleardoublepage
% ---



% ---
% inserir lista de abreviaturas e siglas
% ---
\begin{siglas} %TCC:
  \item[Fig.] Area of the $i^{th}$ component
  \item[456] Isto é um número
  \item[123] Isto é outro número
  \item[Zézão] este é o meu nome
  \item[CCTV] Close-circuit television camera 
\end{siglas}
% ---

%% ---
%% inserir lista de símbolos, se for adequado ao trabalho. %TCC:
%% ---
%\begin{simbolos}
%  \item[$ \Gamma $] Letra grega Gama
%  \item[$ \Lambda $] Lambda
%  \item[$ \zeta $] Letra grega minúscula zeta
%  \item[$ \in $] Pertence
%\end{simbolos}
%% ---

% ---
% inserir o sumario
% ---
\pdfbookmark[0]{\contentsname}{toc}
\tableofcontents*
\cleardoublepage
% ---





% ----------------------------------------------------------
% ELEMENTOS TEXTUAIS
% ----------------------------------------------------------
\textual


% ----------------------------------------------------------
% Introdução
% ----------------------------------------------------------

\chapter[Introdução]{Introdução}
%TCC:
% Contextualização, problema, hipótese, objetivo geral, objetivos específicos, justificativa e resultados esperados.
Em 2019 foi estimado que existem 200 milhões de câmeras de vigilância na China. (\cite{qiang2019road})

\section{Objetivos}

\section{Método}


\chapter{Revisão Bibliográfica}
%TCC:
Um ou mais capítulos (por exemplo, se há duas linhas de trabalhos relacionados).



\chapter{Desenvolvimento}
%TCC:
Um ou mais capítulos (por exemplo um para testes)


\begin{figure}[!ht]
    \centering
	\includegraphics[width=0.55\linewidth]{imagemExemplo.pdf}
	\caption[Isso é o que aparece no sumário]{Imagem de exemplo.}
	\label{fig:graficosVariandoTamanhoRede}
\end{figure}


%TCC:
%TCC:
%TCC:
%TCC:

% ---
% Conclusão
% ---
\chapter[Conclusão]{Conclusão}
%TCC:
E daí?





% ----------------------------------------------------------
% ELEMENTOS PÓS-TEXTUAIS
% ----------------------------------------------------------
\postextual


% ----------------------------------------------------------
% Referências bibliográficas
% ----------------------------------------------------------
\bibliography{abntex2-modelo-references}


%% ----------------------------------------------------------
%% Apêndices TCC: só mantenha se for pertinente.
%% ----------------------------------------------------------

% ---
% Inicia os apêndices
% ---
\begin{apendicesenv}

% Imprime uma página indicando o início dos apêndices
\partapendices

% ----------------------------------------------------------
\chapter{Quisque libero justo}
% ----------------------------------------------------------

\lipsum[50]

% ----------------------------------------------------------
\chapter{Coisas que fiz e que achei interessante mas não tanto para entrar no corpo do texto}
% ----------------------------------------------------------
\lipsum[55-57]

\end{apendicesenv}
% ---


% ----------------------------------------------------------
% Anexos %TCC: so mantenha se pertinente.
% ----------------------------------------------------------

% ---
% Inicia os anexos
% ---
\begin{anexosenv}

% Imprime uma página indicando o início dos anexos
\partanexos

% ---
\chapter{Eu sempre quis aprender latim}
% ---
\lipsum[30]

% ---
\chapter{Coisas que eu não fiz mas que achei interessante o suficiente para colocar aqui}
% ---

\lipsum[31]

% ---
\chapter{Fusce facilisis lacinia dui}
% ---

\lipsum[32]

\end{anexosenv}

%---------------------------------------------------------------------
% INDICE REMISSIVO
%---------------------------------------------------------------------

\printindex



\end{document}